\documentclass{article}
\usepackage[letterpaper, margin=1in]{geometry}
\usepackage{listings}
\usepackage{amsmath}
\usepackage[version=4]{mhchem}
\usepackage{graphicx}
\usepackage[export]{adjustbox}
\begin{document}

\begin{center}
\section*{Numerically Solving the Lane-Emden Equation}
\end{center}
%\begin{center}
%Numerically Solving the Lane-Emden Equation\\
%\end{center}

\begin{itemize}
\item Employing a numerical integrator (see below), compute and compare the density profiles for polytropes $n = 1.5$ (degenerate/fully convective star) and $n = 3$ (star in radiative equilibrium). In both cases, plot the ratio of the density to the central density ($\rho / \rho_c$) as a function of the dimensionless quantity $\xi = r / \lambda_n$. Also compute the values of $\xi_1$ and $dD / d\xi$ at $\xi_1$, where the density goes to zero. Be sure to show both your plot(s) and the code you used to calculate these profiles.
\end{itemize}

Second-order differential equations can be integrated using the Euler method by solving two equations: one for the parameter and one for its derivative. In this case, the Lane-Emden equation can be written as:\\



\begin{center}
\section*{Polytripic Relations}
\end{center}

\begin{itemize}
\item 
\end{itemize}


\end{document}